% Compile Command:
% latex report; bibtex report; latex report; latex report; dvips report; ps2pdf report.ps; open -a Preview report.pdf

% v2-acmsmall-sample.tex, dated March 6 2012
% This is a sample file for ACM small trim journals
%
% Compilation using 'acmsmall.cls' - version 1.3 (March 2012), Aptara Inc.
% (c) 2010 Association for Computing Machinery (ACM)
%
% Questions/Suggestions/Feedback should be addressed to => "acmtexsupport@aptaracorp.com".
% Users can also go through the FAQs available on the journal's submission webpage.
%
% Steps to compile: latex, bibtex, latex latex
%
% For tracking purposes => this is v1.3 - March 2012

\documentclass[prodmode,acmtecs]{acmsmall} % Aptara syntax

% Package to generate and customize Algorithm as per ACM style
\usepackage[ruled]{algorithm2e}
\usepackage{tabularx,placeins}
\renewcommand{\algorithmcfname}{ALGORITHM}
\SetAlFnt{\small}
\SetAlCapFnt{\small}
\SetAlCapNameFnt{\small}
\SetAlCapHSkip{0pt}
\IncMargin{-\parindent}

% Metadata Information
\acmVolume{0}
\acmNumber{0}
\acmArticle{00}
\acmYear{2015}
\acmMonth{3}

% Document starts
\begin{document}

% Page heads
\markboth{S. Paustian}{Technical Communication in Your Field}

% Title portion
\title{Technical Communication in Your Field}
\author{STEVEN PAUSTIAN
\affil{New Mexico Institute of Mining \& Technology}
}
% NOTE! Affiliations placed here should be for the institution where the
%       BULK of the research was done. If the author has gone to a new
%       institution, before publication, the (above) affiliation should NOT be changed.
%       The authors 'current' address may be given in the "Author's addresses:" block (below).
%       So for example, Mr. Abdelzaher, the bulk of the research was done at UIUC, and he is
%       currently affiliated with NASA.

\begin{abstract}
TODO [ the absract goes here, dummy ]
\end{abstract}

\category{C.2.2}{Technical Communication}{Computer Science}

\terms{Comments, Communication, Issue Tracking, Technical Language, Version Control}

\keywords{Technical Communication, ACM Small Format}

\acmformat{Steven Paustian 2015. Technical Communication in Your Field}

\begin{bottomstuff}
This work was intended, designed, and written for ENGL341-02, Spring 2015.

Author's address: S. Paustian, Computer Science Department,
New Mexico Institute of Mining \& Technology, Socorro, NM.
\end{bottomstuff}

\maketitle


\section{Introduction}
Computer Science (CS) is the study of computation and its applications.  CS majors are employed in a wide variety of positions post-graduation; prominent job titles include Application Developer, Information Technology Architect, and Systems Administrator.  \cite{payscale}

The CS field contains a broad array of delivery mechanisms for technical information.  Many are comparable to other fields, including peer reviewed journals, memos, and technical reports.  In addition, it involves a number of communication tools that are unique such as issue tracking and code commenting. 

To further explore technical communication in the field, the author interviewed a recent graduate and a seasoned professional programmer.  A discussion of these interviews can be found in [\ref{interviews}].  Three examples technical documents were gathered and will be discussed in [\ref{ex_documents}].  Two secondary sources regarding technical communication in the field were also found and will be discussed in [\ref{sec_documents}].

% Head 1
\section{Interview Analysis}\label{interviews}

% Head 2
\subsection{Professional Programmer}
Frank Germano, a professional programmer of 30 years, was a wealth of information regarding technical communication.  As a former vice president of a software engineering company, and current lead developer, he noted technical communication requirements may vary widely through one's career. 

As a lowly programmer, most technical communication involves code commenting, software documentation, and issue tracking (verbose descriptions of buggy code).  Lead developers and project managers serve as a bridge between programmers and non-technical management, and their communications reflect this.  They tend to handle state diagrams, flowcharts/UML diagrams, memos, and (sometimes) contracts.  He revealed that larger firms will typically include a Technical Writer in teams working on large project.  This person's sole responsibility is creating, editing and interpreting technical documention for the software the team is currently writing. 

The bulk of a programmer's non-coding time is spent on internal communication, namely emails.

TODO[ Add time limitations, revision process ]

\subsection{Recent Graduate}
Jon Zingale is a recent graduate from The University of Texas as a math major with a CS minor.  He was hired by a security firm in Silicon Valley.  He reports that a large portion of his time is spent on communication; mostly emails, memos, and documenting security problems both in code and an internal issue registration system.  He rarely needs to convey technical information to non-technical personel, though he was recently asked to give a presentation to client regarding security holes discovered by his team.  \\
TODO[ flesh this out ]

\section{Example Documents}\label{ex_documents}
Three example documents were obtained from Frank and Jon.  They are: an example of properly commented code, an example of issue documention, and a monthly memo sent to clients. \\
TODO[ add margins to table, fix table placement, finish table cells ]

\FloatBarrier

{\renewcommand{\arraystretch}{2}
\begin{table}[]
    \centering
    \begin{tabularx}{\textwidth}{ X | X | X | X | X }
        \hline
         Document Type & Audience  & Purpose & Style & Formating \\ \hline

        Issue \newline Tracking & Management tracking progress on a bug fix. \newline \newline Internal programmers assigned to fix the issue.  & Provide specific technical and tracking information regarding a bug that needs to be fixed. & 3 & Conforms to internal specifications designed for clear communication of technical information.  \\ \hline 

        Customer \newline Memo & Current customers and the general public. \newline \newline Company employees. & Provide customers and employees with ``state of the union" information about the company. & 3 & Colorful and easy to read. \newline \newline  Different font weights and colors highlight important information and what it relates too.  \newline \newline Information is regarding new customers and new software functionality.  \\ \hline
        
        Code \newline Comments & Programmers and other employees working directly with the code. & Explicitly explain code parameters, usage, and how the code works. & 3 &  Conforms to internal commenting standards. \\ \hline
    \end{tabularx}
\end{table}


\FloatBarrier

\section{Secondary Documents}\label{sec_documents}
 TODO[ Consider adding brief intro ]
\subsection{Comments Are More Important Than Code}
Communication inside code is unique to programming.  Programmers often talk of making code ``self-commenting,'' but this is nearly impossible to do.  Code is, by definition, complicated and difficult to understand.  Comments are an essential tool for communicating an algorithm's intent, usage, and implementation.  In other words, good comments explain how and why.

Raskin argues that code comments are not only important, but essential for code to be reliable and maintainable.\cite{raskin} The problem is, enforcing strict commenting guidelines is cumbersome and limits the amount of code a programmer can write in a given unit of time.  Raskin argues that despite this, the benefits of rigorous comments far outweight the costs.  Rebuilding code with good documentation is far easier than without.  In fact, writing any code is far simpler if documention exists beforehand. \cite{raskin}

\subsection{Agile Development}
Agile software development is a software development that stresses working code over ideas, and effective person to person communication over documents and long response times.\cite{highsmith}  This section will focus on the communication aspects of Agile development, and how those concepts assist in creating effective and robust software. \\
TODO[ flesh this out ]  

\section{Conclusion}

TODO[ fill this out ]

% Acknowledgments
\begin{acks}
The author would like to thank Frank Germano and Jon Zingale for their interviews.
\end{acks}

% Bibliography
\bibliographystyle{ACM-Reference-Format-Journals}
\bibliography{acmsmall-sample-bibfile}
                             % Sample .bib file with references that match those in
                             % the 'Specifications Document (V1.5)' as well containing
                             % 'legacy' bibs and bibs with 'alternate codings'.
                             % Gerry Murray - March 2012

\end{document}
% End of v2-acmsmall-sample.tex (March 2012) - Gerry Murray, ACM



% Compile Command:
% latex report; bibtex report; latex report; latex report; dvips report; ps2pdf report.ps; open -a Preview report.pdf

% v2-acmsmall-sample.tex, dated March 6 2012
% This is a sample file for ACM small trim journals
%
% Compilation using 'acmsmall.cls' - version 1.3 (March 2012), Aptara Inc.
% (c) 2010 Association for Computing Machinery (ACM)
%
% Questions/Suggestions/Feedback should be addressed to => "acmtexsupport@aptaracorp.com".
% Users can also go through the FAQs available on the journal's submission webpage.
%
% Steps to compile: latex, bibtex, latex latex
%
% For tracking purposes => this is v1.3 - March 2012

\documentclass[prodmode,acmtecs]{acmsmall} % Aptara syntax

% Package to generate and customize Algorithm as per ACM style
\usepackage[ruled]{algorithm2e}
\renewcommand{\algorithmcfname}{ALGORITHM}
\SetAlFnt{\small}
\SetAlCapFnt{\small}
\SetAlCapNameFnt{\small}
\SetAlCapHSkip{0pt}
\IncMargin{-\parindent}

% Metadata Information
\acmVolume{0}
\acmNumber{0}
\acmArticle{00}
\acmYear{2015}
\acmMonth{3}

% Document starts
\begin{document}

% Page heads
\markboth{S. Paustian}{Technical Communication in Your Field}

% Title portion
\title{Technical Communication in Your Field}
\author{STEVEN PAUSTIAN
\affil{New Mexico Institute of Mining \& Technology}
}
% NOTE! Affiliations placed here should be for the institution where the
%       BULK of the research was done. If the author has gone to a new
%       institution, before publication, the (above) affiliation should NOT be changed.
%       The authors 'current' address may be given in the "Author's addresses:" block (below).
%       So for example, Mr. Abdelzaher, the bulk of the research was done at UIUC, and he is
%       currently affiliated with NASA.

\begin{abstract}
TODO: ABSTRACT
\end{abstract}

\category{C.2.2}{Technical Communication}{Computer Science}

\terms{Comments, Communication, Issue Tracking, Technical Language, Version Control}

\keywords{Technical Communication, ACM Small Format}

\acmformat{Steven Paustian 2015. Technical Communication in Your Field}
% At a minimum you need to supply the author names, year and a title.
% IMPORTANT:
% Full first names whenever they are known, surname last, followed by a period.
% In the case of two authors, 'and' is placed between them.
% In the case of three or more authors, the serial comma is used, that is, all author names
% except the last one but including the penultimate author's name are followed by a comma,
% and then 'and' is placed before the final author's name.
% If only first and middle initials are known, then each initial
% is followed by a period and they are separated by a space.
% The remaining information (journal title, volume, article number, date, etc.) is 'auto-generated'.

\begin{bottomstuff}
This work was intended, designed, and written for ENGL341-02, Spring 2015.

Author's address: S. Paustian, Computer Science Department,
New Mexico Institute of Mining \& Technology, Socorro, NM.
\end{bottomstuff}

\maketitle


\section{Introduction}

TODO: INTRODUCTION

% Head 1
\section{Interviews}

% Head 2
\subsection{Professional 1}
TODO
\subsection{Professional 2}
TODO

\section{Typical references in new ACM Reference Format}
TODO

% Appendix
\appendix
\section*{APPENDIX}
\setcounter{section}{1}
In this appendix, we measure
the channel switching time of Micaz [CROSSBOW] sensor devices.
In our experiments, one mote alternatingly switches between Channels
11 and 12. Every time after the node switches to a channel, it sends
out a packet immediately and then changes to a new channel as soon
as the transmission is finished. We measure the
number of packets the test mote can send in 10 seconds, denoted as
$N_{1}$. In contrast, we also measure the same value of the test
mote without switching channels, denoted as $N_{2}$. We calculate
the channel-switching time $s$ as
\begin{eqnarray}%
s=\frac{10}{N_{1}}-\frac{10}{N_{2}}. \nonumber
\end{eqnarray}%
By repeating the experiments 100 times, we get the average
channel-switching time of Micaz motes: 24.3$\mu$s.

\appendixhead{ZHOU}

% Acknowledgments
\begin{acks}
The authors would like to thank Dr. Maura Turolla of Telecom
Italia for providing specifications about the application scenario..
\end{acks}

% Bibliography
\bibliographystyle{ACM-Reference-Format-Journals}
\bibliography{acmsmall-sample-bibfile}
                             % Sample .bib file with references that match those in
                             % the 'Specifications Document (V1.5)' as well containing
                             % 'legacy' bibs and bibs with 'alternate codings'.
                             % Gerry Murray - March 2012

\end{document}
% End of v2-acmsmall-sample.tex (March 2012) - Gerry Murray, ACM


